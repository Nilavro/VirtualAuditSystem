\documentclass[sigconf]{acmart}

\usepackage{booktabs} % For formal tables


% Copyright
%\setcopyright{none}
%\setcopyright{acmcopyright}
%\setcopyright{acmlicensed}
\setcopyright{rightsretained}
%\setcopyright{usgov}
%\setcopyright{usgovmixed}
%\setcopyright{cagov}
%\setcopyright{cagovmixed}


% DOI
%\acmDOI{10.475/123_4}

% ISBN
%\acmISBN{123-4567-24-567/08/06}

%Conference
%\acmConference[WOODSTOCK'97]{ACM Woodstock conference}{July 1997}{El Paso, Texas USA} 
%\acmYear{1997}
%\copyrightyear{2016}

%\acmPrice{15.00}


\begin{document}
\title{New Directions Papers}
\titlenote{Produces the permission block, and
  copyright information}
\subtitle{Extended Abstract}
\subtitlenote{The full version of the author's guide is available as
  \texttt{acmart.pdf} document}


\author{Nilavra Pathak}
\orcid{1234-5678-9012}
\affiliation{%
  \institution{University of Maryland Baltimore County}
  \streetaddress{1000 Hilltop Circle}
  \city{Baltimore} 
  \state{Maryland} 
  \postcode{21250}
}
\email{nilavra1@umbc.edu}

 

 

\begin{abstract}
In addition to regular papers on the topics above, ACM SenSys this year solicits new directions papers discussing the future of sensor-enabled computer systems. New direction papers shall present a compelling vision of the field five years from now, discuss what current or foreseeable technologies hold the highest potential within this time frame, and argue on what are the key research problems the community needs to tackle right now to eventually realize this vision. New direction papers will be reviewed separately by the SenSys steering committee. Accepted papers will be part of the proceedings, alongside regular papers, and their authors will be invited to a panel session, together with leading experts, to share their vision."New directions" papers are required to follow the same submission guidelines, but be limited to 6 pages everything included.\footnote{This is an abstract footnote}
\end{abstract}

%
% The code below should be generated by the tool at
% http://dl.acm.org/ccs.cfm
% Please copy and paste the code instead of the example below. 
%
\begin{CCSXML}
<ccs2012>
 <concept>
  <concept_id>10010520.10010553.10010562</concept_id>
  <concept_desc>Computer systems organization~Embedded systems</concept_desc>
  <concept_significance>500</concept_significance>
 </concept>
 <concept>
  <concept_id>10010520.10010575.10010755</concept_id>
  <concept_desc>Computer systems organization~Redundancy</concept_desc>
  <concept_significance>300</concept_significance>
 </concept>
 <concept>
  <concept_id>10010520.10010553.10010554</concept_id>
  <concept_desc>Computer systems organization~Robotics</concept_desc>
  <concept_significance>100</concept_significance>
 </concept>
 <concept>
  <concept_id>10003033.10003083.10003095</concept_id>
  <concept_desc>Networks~Network reliability</concept_desc>
  <concept_significance>100</concept_significance>
 </concept>
</ccs2012>  
\end{CCSXML}

\ccsdesc[500]{Computer systems organization~Embedded systems}
\ccsdesc[300]{Computer systems organization~Redundancy}
\ccsdesc{Computer systems organization~Robotics}
\ccsdesc[100]{Networks~Network reliability}

% We no longer use \terms command
%\terms{Theory}

\keywords{ACM proceedings, \LaTeX, text tagging}


\maketitle

 
\bibliographystyle{ACM-Reference-Format}
\bibliography{sigproc} 

\end{document}
